\chapter{Introduction}
\label{cha:intro}
\section{Background}
\label{sec:pre_aim}
\section{Motivations and Aims}
\label{sec:aim}
To mimic the vision processing in the brain.

Exploring and mimicking invariant object recognition within the brain is a promising approach to tackling the computational difficulty;
in turn it also contributes to understanding biological visual processing by means of mimicking neural activity in the visual system of the brain.
Moreover, energy-efficiency improvements following from the great energy efficiency of biological systems will help in building object recognition systems, e.g. posture recognition for human-machine interfaces in mobile devices.  
%TODO How
Basic SNN description and H/W set-up.\\
Modelling the network with populations of neurons, synapses and learning rules.

\section{Contributions}
\label{sec:ctb}

\section{Publications}
Real-time recognition of dynamic hand postures on a neuromorphic system. Chapter-3

Benchmarking Spike-Based Visual Recognition: a Dataset and Evaluation. Chapter-3, 6

Noisy Softplus: A Biology Inspired Activation Function. Chapter-4

STDP Training on Rate-based Spiking Autoencoders and RBMs. Chapter-5

Modeling  Populations  of  Spiking  Neurons  for  Fine  Timing  Sound Localization.
\section{Thesis Structure}
\label{sec:str}