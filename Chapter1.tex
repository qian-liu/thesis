\chapter{Introduction}
\label{cha:intro}

%TODO problem, motivation and significance
Tell the story: how I find the problem

Why it motivates me (why it would be useful and important to have a solution)

Some background knowledge about SpiNNaker, DVS and plasticity.


\section{Statement of The Problem}
\label{sec:problem}
There is a gap between the recognition capability and learning ability between SNNs and state-of-the-art ANNs. %detailed proofs should be listed in Chapter 2 Literature review.


\section{Hypotheses and Aims}
\label{sec:aim}
\begin{itemize}
	\item 
	An object recognition system can operate in real-time on a complete neuromorphic platform in an absolute spike-based, energy-efficient fashion.
	The hypothesis pave the way for further study with solid believe of low-cost and low latency cognitive application built on neuromorphic platform.

	Aim: to build a real-time hand posture recognition system on the hardware SNN simulator, SpiNNaker, receiving visual input from a DVS sensor.

	\item 
	SNNs can deliver equivalent cognitive capability as conventional ANNs for object recognition applications.

	Aim: to generalise a training method on conventional ANNs whose trained connections can be applied to corresponding SNNs with close recognition performance.

	\item 
	SNNs can be trained with biologically-realistic synaptic plasticity and demonstrate competent learning capability as ANNs, even as state-of-the-art deep architectures.

	Aim: to formalise a local learning algorithm based on synaptic plasticity for training deep SNNs.

	\item 
	A new set of spike-based vision datasets can provide resources to support fair competition between researchers since new concerns on energy efficiency and recognition latency emerge in Neuromorphic Vision.

	Aim: to provide a unified spiking version of common-used dataset and complementary evaluation methodologies to assess the performance of SNN algorithms.
\end{itemize}


\section{Contributions}
\begin{itemize}
	\item An implementation of 
\end{itemize}
\section{Thesis Structure}
\section{Publications and Workshops}
\subsection{Contribute to The Thesis}
\begin{itemize}
	\item 
	\textbf{Q. Liu}, and S. Furber, “Real-Time Recognition of Dynamic Hand Postures on a Neuromorphic System”, International Conference on Artificial Neural Networks (ICANN 2015)
	
	\item 
	\textbf{Q. Liu}, G. Garcis, E. Stromatias, T. Gotarredona, and S. Furber, “Benchmarking Spike-Based Visual Recognition: A Dataset and Evaluation ,” Frontiers in Neuromorphic Engineering. Chapter-6
	
	\item 
	\textbf{Q. Liu}, and S. Furber, “Noisy Softplus: A Biology Inspired Activation Function”, International Conference on Neural Information Processing (ICONIP 2016) Chapter-4
	
	\item 
	\textbf{Q. Liu}, and S. Furber, “STDP Training on Rate-Based Spiking Autoencoders”, International Joint Conference on Neural Network (to submit to IJCNN 2017 ). Chapter-5
\end{itemize}

\subsection{Excluded on The Thesis}
\begin{itemize}
	\item 
	G. Garcis, P. Camilleri, \textbf{Q. Liu}, and S. Furber, “pyDVS: A real-time dynamic vision sensor emulator using off-the-shelf hardware”, The 2016 IEEE Symposium Series on Computational Intelligence (IEEE SSCI 2016).
	
	\item
	\textbf{Q. Liu}, C. Patterson, S. Furber, Z. Huang, Y. Hou and H. Zhang, “Modeling Populations of Spiking Neurons for Fine Timing Sound Localization”, International Joint Conference on Neural Networks (IJCNN 2013)
\end{itemize}	

\subsection{Workshops}
It is essential for the author to participate in the workshops organised by the close community, 1) to establish and contribute to collaborations on mutual interests; 2) to catch up with the cutting-edge research and collect inspiration; 3) to discuss the author's own findings with the key researchers in the field.
\begin{itemize}
	\item 
	\textit{Capo Caccia Cognitive Neuromorphic Engineering Workshop 2012}.
	
	Contributed to successfully connections of SpiNNaker to neuromorphic sensors\footnote{\url{https://capocaccia.ethz.ch/capo/wiki/2012/csnQian}}. 
	It formed the hardware platform for real-time SNN applications processing event-based sensor data.
	
	\item 
	\textit{Telluride Neuromorphic Cognition Engineering Workshop 2013}.
	
	Developed a real-time sound localisation system on the neuromorphic platform as a main contributor\footnote{\url{http://neuromorphs.net/nm/wiki/sound_localization}}.
	The work led to the publication of a journal paper\cite{lagorce2015breaking}.
	
	
	\item 
	\textit{Capo Caccia Cognitive Neuromorphic Engineering Workshop 2014}.
	
	Developed the real-time neural activity visualiser for the project of ``Integrated Neurorobotics for Real-World Cognitive Behaviour'' \footnote{\url{https://capocaccia.ethz.ch/capo//wiki/2014/integrneurobot14}}. 
	
	\item 
	\textit{Capo Caccia Cognitive Neuromorphic Engineering Workshop 2015}.
	
	Fully inspired by the projects of vision tasks in the workshop, the author later proposed the translation methods and the unsupervised learning algorithm of spiking deep networks.
	And led the discussion of benchmarking neuromorphic vision in the workshop \footnote{\url{https://capocaccia.ethz.ch/capo//wiki/2015/visionbenchmark15}}. 	
\end{itemize}




