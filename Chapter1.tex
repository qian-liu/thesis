\chapter{Introduction}
\label{cha:intro}

%TODO problem, motivation and significance
Tell the story: how I find the problem

Why it motivates me (why it would be useful and important to have a solution)

Some background knowledge about SpiNNaker, DVS and plasticity.


\section{Statement of The Problem}
\label{sec:problem}
There is a gap between the recognition capability and learning ability between SNNs and state-of-the-art ANNs. %detailed proofs should be listed in Chapter 2 Literature review.


\section{Hypotheses and Aims}
\label{sec:aim}
An object recognition system can operate in real-time on a complete neuromorphic platform in an absolute spike-based, energy-efficient fashion.
The hypothesis pave the way for further study with solid believe of low-cost and low latency cognitive application built on neuromorphic platform.
Aim: to build a real-time hand posture recognition system on the hardware SNN simulator, SpiNNaker, receiving visual input from a DVS sensor.

SNNs can deliver equivalent cognitive capability as conventional ANNs for object recognition applications.
Aim: to generalise a training method on conventional ANNs whose trained connections can be applied to corresponding SNNs with close recognition performance.

SNNs can be trained with biologically-realistic synaptic plasticity and demonstrate competent learning capability as ANNs, even as state-of-the-art deep architectures.
Aim: to formalise a local learning algorithm based on synaptic plasticity for training deep SNNs.

A new set of spike-based vision datasets can provide resources to support fair competition between researchers since new concerns on energy efficiency and recognition latency emerge in Neuromorphic Vision.
Aim: to provide a unified spiking version of common-used dataset and complementary evaluation methodologies to assess the performance of SNN algorithms.




\section{Contributions}
\section{Thesis Structure}
\section{Publications}
Q. Liu, and S. Furber, “Real-Time Recognition of Dynamic Hand Postures on a Neuromorphic System”, International Conference on Artificial Neural Networks (ICANN 2015)

Q. Liu, G. Garcis, E. Stromatias, T. Gotarredona, and S. Furber, “Benchmarking Spike-Based Visual Recognition: A Dataset and Evaluation ,” Frontiers in Neuromorphic Engineering. Chapter-6

Q. Liu, and S. Furber, “Noisy Softplus: A Biology Inspired Activation Function”, International Conference on Neural Information Processing (ICONIP 2016) Chapter-4

Q. Liu, and S. Furber, “STDP Training on Rate-Based Spiking Autoencoders”, International Joint Conference on Neural Network (to submit to IJCNN 2017 ). Chapter-5

G. Garcis, P. Camilleri, Q. Liu, and S. Furber, “pyDVS: A real-time dynamic vision sensor emulator using off-the-shelf hardware”, The 2016 IEEE Symposium Series on Computational Intelligence (IEEE SSCI 2016).

Q. Liu, C. Patterson, S. Furber, Z. Huang, Y. Hou and H. Zhang, “Modeling Populations of Spiking Neurons for Fine Timing Sound Localization”, International Joint Conference on Neural Networks (IJCNN 2013)